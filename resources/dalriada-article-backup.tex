\documentclass{article}

\def\signed #1{{\leavevmode\unskip\nobreak\hfil\penalty50\hskip2em
  \hbox{}\nobreak\hfil(#1)%
  \parfillskip=0pt \finalhyphendemerits=0 \endgraf}}

\newsavebox\mybox
\newenvironment{aquote}[1]
  {\savebox\mybox{#1}\begin{quote}}
  {\signed{\usebox\mybox}\end{quote}}

\title{Celebrating Excellence Kelsey McKenna 1st Place in A Level Maths 2014}
\date{22nd April 2015}
\author{Compiled by Mrs P Crerand}

\begin{document}
\maketitle

\begin{aquote}{Mrs Scott}
I first came across Kelsey when he came to my class one day to recite pi to a great number of decimal places. My pupils were so impressed. The next time I saw Kelsey performing he was solving a Rubik’s Cube on the stage in the Riverside Euphoria concert. I always associate Kelsey as someone who enjoyed Maths. I never taught Kelsey but I knew of his great talent for Maths from an early age. Kelsey was an outstanding Mathematician who gained first place in Northern Ireland in 2014 in A Level Mathematics. This is a huge achievement and requires great dedication and outstanding ability. Kelsey had it all. I have put together a few comments from some of his teachers:

Kelsey has been a star from an early age. He has always been his own person and had that intellectual quirkiness which is very much in keeping with the renowned Dalriada spirit. He loves knowledge for knowledge’s sake. This became apparent to me when he took an interest in the number “pi”. Others were writing poems, dances, music and eating to celebrate pi day but Kelsey decided to start his love affair with pi by memorising it to as many digits as he could. Before too long he passed 20 decimal places, then 30, then 100, then 150, then 200, then 220. No doubt he has gone to many more decimal places by now. Why would he do this? Some people memorise pi to show off, but I believe that for Kelsey he did it to gain a profound understanding of randomness and large numerals in general. Saying that pi is 3.14 without addressing the fact that it goes on infinitely afterwards overlooks the true essence of the number. To him, pi represented the beauty of Mathematics, the beauty of the world we live in. This unattainable, irrational, transcendental number is used to describe the perfect symmetry of circles and spheres. Memorizing pi is a step forward in the search for a pattern in pi, and in general, for finding regularity in chaos. Mathematics searches for logical rules, but pi seems to have a logic of its own. This interested a very young Kelsey and demonstrated just how special and gifted he is, as well as being a delightful and humble young man. Full of fun, he was always up for performing his pi party trick to my classes. What a privilege to know him.
\end{aquote}

\begin{aquote}{Mrs Cousins}
Kelsey is an extremely gifted, talented and determined mathematician. I found him to be very inquisitive with a real thirst for Mathematical knowledge, and he regularly independently went above and beyond the syllabus to further his knowledge, and gain a fuller understanding of Mathematical concepts. He was a pleasure to teach and I wish him every success in the future.
\end{aquote}

\begin{aquote}{Dr McIvor}
Kelsey is easily the most talented pupil I have taught at A Level; his work ethic, attention to detail and determination to succeed are second to none. As a result, he was always going to achieve an A* grade but I can remember saying to Kelsey and his mother at the year 14 parents evening that he should aim to come first in Northern Ireland; he certainly delivered in style! Watch this space as I firmly believe that we will be hearing a lot more about Kelsey in years to come. I wish him every success in the future.
\end{aquote}

\begin{aquote}{Mr T Gamble}
From a computer science perspective, Kelsey just couldn’t get enough. I remember a particular problem that the class was studying in year 13 called the ‘jailer-hat’ problem. The next day, after class, Kelsey eagerly showed me his own solution algorithm that he turned into a program that could solve the problem instantly. Here was boy with a computing-brain, a gifted problem-solver and rational thinker that could go beyond his years to formulate some of the finest coding I have and will ever see from a sixth former. Take for instance his final year project – a Rubik’s cube solver. This program, spanning over 1000 pages of documentation, could solve any cube state in the minimum amount of turns. AQA had no hesitation in agreeing with us that this project was something special. So, without surprise, we received news later on to confirm that Kelsey was placed in the top 4 pupils in the UK for A-level computer science. But let me tell you something far more impressive about Kelsey; something more memorable and inspiring. In spite of achievements and gifting, Kelsey will always be one of the most humble pupils I have ever taught. Now that’s amazing.
\end{aquote}

\end{document}
